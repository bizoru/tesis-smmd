\chapter{Centros de Computación de Alto desempeño}

\section{Centros de Computación de Alto desempeño}
Un centro de computación de alto desempeño es una colección de computadoras denominadas nodos las cuales se encuentran interconectadas a través de una red de alta velocidad establecidas en un arquitectura específica, dichos nodos se clasifican según las tareas que se desarrollen en estos. Una de las principales motivaciones en el uso de Centros de Computación de alto desempeño, es la paralelización de algoritmos para la obtención del mejor desempeño, reduciendo tiempo de computo haciendo y haciendo un uso eficiente de los recursos en el procesamiento de grandes volúmenes de información.  
\\\\
En el área de la computación de alto desempeño se abarcan los campos científicos y técnicos en el estudio de las supercomputadoras \cite{nielsen_2016} , el sitio Top 500 \cite{top500_supercomputer_sites_2019} mantiene una lista actualizada de las quinientas mejoras computadoras en el mundo y las clasifica según su rendimiento, este es medido en FLOPS\footnote{Floating point operations per second, Cantidad de operaciones aritméticas de punto flotante por segundo}, para Noviembre de 2018 la supercomputadora número 1 fue Power System AC922 de IBM con 2,397,824 núcleos de procesador y una memoria RAM de 2,801,664 Gigabytes alcanzando un máximo teórico de rendimiento computacional de 200,795 Tera flops por segundo.



\newpage

\section{CECAD, Centro de Computación de Alto Desempeño de la Universidad Distrital Francisco José de Caldas}

El centro de computación de alto desempeño de la Universidad Distrital Francisco José de Caldas, es un laboratorio destinado a la investigación en donde ser realizan distintos experimentos y pruebas, este es un espacio destinado a la investigación y el aprendizaje, allí se realizan pruebas de nuevas tecnología y se desarrollan distintos proyectos de investigación provenientes de los grupos de investigación de pregrado, maestría y doctorado.\cite{caliz_rodolfo_2019}.
\\
La estructura general del CECAD está compuesta por la siguiente infraestructura: 

\begin{enumerate}
	\item Un cluster HPC homogéneo 
	\item Cluster HPC heterógeneo respaldado por 1 nodo GPU con una tarjeta NVIDIA Telsa K-80 y 4 Servidores Intel Xeon Phi z200 7230.
	\item Nube privada respaldada 36 equipos Dell R610, 4 equipos Dell R900 con un total de almacenamiento de 60 TB.
	\item Nube privada de pruebas, esta nube se encuentra basada en openstackl, la cual se utilizara para probar nuevos servicios y actualizaciones.
	\item Distema de archivos distribuido
	\item Sistema de Respaldo implementado a través de un SAN Controller service.
\end{enumerate}